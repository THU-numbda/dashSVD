%% Overleaf			
%% Software Manual and Technical Document Template	
%% 									
%% This provides an example of a software manual created in Overleaf.

\documentclass{ol-softwaremanual}

% Packages used in this example
\usepackage{graphicx}  % for including images
\usepackage{microtype} % for typographical enhancements
\usepackage{minted}    % for code listings
\usepackage{amsmath}   % for equations and mathematics
\setminted{style=friendly,fontsize=\small}
\usepackage{hyperref}  % for hyperlinks
\usepackage[a4paper,top=4.2cm,bottom=4.2cm,left=3.5cm,right=3.5cm]{geometry} % for setting page size and margins
\usepackage{listings}

\lstset{
    columns=fixed,       
    numbers=none,                                      frame=none,
    backgroundcolor=\color[RGB]{244,244,244},          keywordstyle=\color[RGB]{40,40,255},
    numberstyle=\footnotesize\color{darkgray},
    commentstyle=\it\color[RGB]{0,96,96},
    stringstyle=\rmfamily\slshape\color[RGB]{128,0,0},
    showstringspaces=false,
    language=C,
    basicstyle=\small,
    breaklines=true
}

% Custom macros used in this example document
\newcommand{\doclink}[2]{\href{#1}{#2}\footnote{\url{#1}}}
\newcommand{\cs}[1]{\texttt{\textbackslash #1}}
\newcommand{\atn}[1]{\textcolor{black}{#1}}
\newcommand{\atnn}[1]{\textcolor{black}{#1}}
\newcommand{\notice}[1]{\textcolor{blue}{#1}}


\newcommand{\cmtt}{\fontfamily{cmtt}\selectfont}

% Frontmatter data; appears on title page
\title{User Manual for dashSVD: A Faster Randomized SVD Algorithm with Dynamic Shifts for Sparse Data}
\author{Xu Feng, Wenjian Yu}
\date{March, 2024}

\begin{document}

\maketitle

\tableofcontents
\newpage

\section{Introduction}
This is the user manual of the programs of dashSVD algorithm with shifted power iteration for sparse data. The programs are implemented both in Matlab and in C. The details of algorithm are included in the relevant paper.

\section{Efficient Implementation of dashSVD}
In this section, we first briefly introduce \atnn{the Math Kernel Library (MKL)}~\cite{Intel}, which is an efficient library for matrix computations.
%the computation of the matrix. 
Then, we present the details of dashSVD (Alg. 5) \atnn{program written in C based on MKL}. Finally, we provide the \atnn{program interface in} both C and Matlab.

\subsection{Overview of MKL}

\atnn{MKL is developed} by Intel for high-performance \atnn{implementation of linear algebra computation}~\cite{Intel}. MKL not only provides \atnn{the} parallel implementations {of functions in BLAS \cite{BLAS} and LAPACK \cite{LAPACK}}, 
but also supports \atnn{some sparse matrix related} computations \atnn{, like the multiplication of a sparse matrix} and a dense matrix. Therefore, \atn{based on} MKL, we can implement dashSVD to obtain a C program with \atnn{good} parallel efficiency. 

\subsection{\atnn{Implementation Details of dashSVD}}
To improve the performance of dashSVD, we implement it in C language with MKL and multi-thread computing. Several careful treatments are imposed in the implementation for better performance on time and parallel efficiency with less memory usage. 

Firstly, we reduce the peak memory usage by \atnn{allocating} the space of matrices when they are computed
%at the beginning of dashSVD 
and freeing the space of matrices immediately when they are \atnn{not needed}. Secondly, the following treatments are applied \atn{to implement dashSVD with better parallelization.}
\begin{itemize}
	\item We use \atnn{the high-performance parallel implementations of matrix operations} in MKL as much as possible. Therefore, the matrix-matrix, matrix-vector and vector-vector operations are naturally parallelized. We use the following functions of MKL/LAPACK/BLAS to do the main computations in Alg.~5.%, which is also shown in the workflow of Alg.~5 in \atn{XXX}.
	\begin{itemize}
		\item \atn{\texttt{mkl\_sparse\_d\_mm} : Compute the multiplication of a sparse matrix stored in \atnn{compressed sparse row (CSR)} format with a dense matrix, which is mainly used in Step~2, 4 and 9 in Alg.~5.}  
		\item \texttt{cblas\_dgemm} : Compute the matrix-matrix multiplication of two dense matrices, which is mainly used in ``eigSVD'' \atnn{(Alg. 3) invoked} in Step 2, 4 and 10 \atnn{of Alg. 5}, and the computation of $\mathbf{V}$ in Step 11 \atnn{of Alg. 5}.
		\item \texttt{LAPACKE\_dsyevd} : Compute the EVD of a dense matrix, which is used in  \atnn{``eigSVD''}.
	\end{itemize}
	\item We parallelize the copy operations of matrix and vectors with OpenMP following the implementation in \cite{martinsson2016randomized2} \cite{randqb-code}. Therefore, all copy operations in dashSVD are well parallelized with less runtime.
	%Because many copy operations of matrix and vectors are involved in Alg.~3, we parallelize them well with OpenMP to reduce the runtime.
\end{itemize}



\subsection{Interface of dashSVD}

In this subsection, we describe the most important functionality and features of dashSVD using the C interface. Firstly, the sparse matrix should be read in as {\cmtt mat\_csr} structure, \atnn{which includes  the following members}.
\begin{itemize}
	\item {\cmtt nnz}, {\cmtt nrows} and {\cmtt ncols}: the \atn{numbers} of nonzeros, rows and columns of the \atn{sparse} matrix.
	\item {\cmtt values}: the array \atnn{of length {\cmtt nnz} storing the values of nonzero elements}.
	\item {\cmtt cols}: the array \atnn{of length {\cmtt nnz} storing the column indices of nonzero elements}. The column index \atnn{ranges} from 1 to {\cmtt ncols}.
    \item \atn{{\cmtt pointerI}: the array \atnn{of} length {\cmtt nrows}+1 storing the indices of each \atnn{row's starting position} in {\cmtt cols}. For example, \atnn{the column indices of nonzero elements in the $i$-th row are stored as {\cmtt cols[{pointerI}[i-1]]} through {\cmtt cols[{pointerI}[i]]}}.}
\end{itemize}
Parameters {\cmtt cols}, {\cmtt pointerI} %and {\cmtt pointerE} 
are designed following \atnn{the CSR} format in MKL. Besides, the structure {\cmtt mat} is used to store the dense matrix in column major format, \atnn{including the following members}.
\begin{itemize}
	\item {\cmtt nrows} and {\cmtt ncols}: the \atn{numbers} of rows and columns of the \atn{dense} matrix.
	\item {\cmtt d}: the array storing the value of matrix in size {\cmtt nrows} $\times$ {\cmtt ncols} in column-major format.
\end{itemize}

After specifying the input sparse matrix stored in CSR format, we can call the \atnn{following functions to run the dashSVD algorithm}:

\indent\indent{\cmtt dashSVD(A, U, S, V, k)}\\
\indent\indent{\cmtt dashSVD\_opt(A, U, S, V, k, p, s, tol)}\\
%{\cmtt dashSVD} is the interface of Alg.~4 while {\cmtt dashSVD\_pve} is the interface of Alg.~5. 
\atnn{Here,} {\cmtt A} is the input matrix \atnn{in} structure {\cmtt mat\_csr}. {\cmtt U}, {\cmtt S} and {\cmtt V} are the matrices \atnn{in} structure {\cmtt mat}\atnn{. Their dimensions are {\cmtt A.nrows} $\times$ {\cmtt k}, {\cmtt k} $\times$ 1 and {\cmtt A.ncols} $\times$ {\cmtt k}, respectively.} {\cmtt k} is the target rank of truncated SVD. \atnn{Function {\cmtt dashSVD\_opt} provides an option for user to specify some parameters, where the arguments} {\cmtt p}, {\cmtt s} and {\cmtt tol} are $p_{max}$, $s$ and tol in Alg.~5 \atnn{respectively}. \atnn{For function} {\cmtt dashSVD}, the \atnn{values of} {\cmtt p}, {\cmtt s} and {\cmtt tol} are \atnn{fixed, i.e.} 1000, {\cmtt k}/2 and $10^{-2}$\atnn{, respectively}. 

\atnn{We} also provide the interface of dashSVD in Matlab:

\indent\indent{\cmtt [U, S, V] = dashSVD(A, k, p, s, tol)}\\
\atnn{Here, the arguments} {\cmtt p}, {\cmtt s} and {\cmtt tol} can be \atnn{blank. In that case, the default values  1000, {\cmtt k}/2 and $10^{-2}$ are used}.

\section{Source Files for the Main algorithms and Drivers}
\subsection{Main Algorithms}
\begin{itemize}

\item \texttt{Src/matlab/exxamples/basic\_rSVD\_shift.m}: basic randomized SVD algorithm with shifted power iteration and shift updating scheme (Alg. 2 without accelerating skills).

\item \texttt{Src/matlab/src/dashSVD.m}: dashSVD with shifted power iteration for sparse data in Matlab (dashSVD with accelerating skills).

\item \texttt{Src/mkl/src/dashsvd.c}: dashSVD algorithm in our paper which is implemented in C with MKL and OpenMP.
\end{itemize}

\subsection{Drivers in Matlab}
\begin{itemize}
\item \texttt{Src/matlab/example/ShiftedPowerIteration\_Test.m}: used to test the effectiveness of shifted power iteration. The comparison is between Alg. 1 (\texttt{basic\_rSVD.m})~\cite{Halko2011Finding}, Alg. 2* (\texttt{basic\_rSVD\\\_shift\_noupdate.m}) and Alg. 2 (\texttt{basic\_rSVD\_shift.m}) on Dense1 or Dense2.
\item \texttt{Src/matlab/example/dashSVD\_Test.m}: used to test the effectiveness of dashSVD compared with LanczosBD in \texttt{svds} in Matlab on SNAP. (Notice that \texttt{svdstest.m} and \texttt{LanczosBDtest.m} is the modification of \texttt{svds} to make sure LanczosBD can produce results with settled times of restarting.)
\item \texttt{Src/matlab/example/AccuracyControl\_Test.m}: used to validate the PVE criterion of accuracy control used in dashSVD.
\end{itemize}

\subsection{Drivers in C}

The program for testing dashSVD with MKL is in "\texttt{Src/mkl/example}". The MKL library needs the support of oneAPI in Intel~\cite{Intel}. When the libraries MKL and OpenMP have been installed, firstly modified the path of MKL in the makefile("\texttt{Src/mkl/examples/makefile}"), and secondly use "make" 
\begin{lstlisting} 
>> make
\end{lstlisting}
to produce the executable program "dashsvdtest". Developers can use
\begin{lstlisting} 
>> OMP_NUM_THREADS=8 OMP_PROC_BIND=close OMP_PLACES=cores ./dashsvdtest
\end{lstlisting}
to run the driver with one NUMA domain for comparison in one socket with 8 threads. The result of program is an example of testing the dataset SNAP~\cite{snapnets} on dashSVD with and without accuracy control.
\section{Additional Notes}
\begin{itemize}
\item Because the source codes in C rely on not only the general functions in BLAS and LAPACK but also the function {
\cmtt mkl\_sparse\_d\_mm
} to compute the multiplication of a sparse matrix stored in CSR format with a dense matrix only in MKL, they are not able to be complied directly with other versions of BLAS and LAPACK.
\item When compiled on MacOS, one should download the older version of oneAPI to install MKL, because from oneAPI version 2024 on macOS is not supported. Besides,  OpenMP should be prepare when the compile clang dose not support OpenMP. Then, use clang instead of gcc as the compiler to compile the source codes with correct path of links. 
\end{itemize}


	\bibliographystyle{acm}
	\bibliography{icml22}


\end{document}
